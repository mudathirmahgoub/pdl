\documentclass[12pt,a4paper]{article}
\usepackage[utf8]{inputenc}
\usepackage{amsmath}
\usepackage{amsfonts}
\usepackage{amssymb}
\usepackage{makeidx}
\usepackage{graphicx}
\usepackage{lmodern}
\usepackage{url}
\usepackage{bm}
\usepackage{booktabs}
\usepackage[left=2cm,right=2cm,top=2cm,bottom=2cm]{geometry}
\author{Mudathir Mahgoub}
\title{Project report}
\begin{document}

\maketitle

\section{Description of PDL solver}
This PDL solver checks the satisfiability of formulas in propositional dynamic logic (PDL).
The input is a kripke frame, which is optional, followed by a formula in PDL. The output is the satisfiability result of this formula which is either \textit{unsat}, \textit{sat} or \textit{unknown}. If the result is \textit{sat}, the solver outputs a kripke frame that satisfies the formula. Furthermore, it specifies the set of states where the formula is satisfied. If this set is equal to $K$, the set of all states in the kripke frame, then the formula is valid in this kripke frame. Otherwise, this set satisfies the formula and its complement, with respect to $K$, falsifies the formula. 

To check the validity of a formula in all kripke frames, its negation should be used as an input and if the result is \textit{unsat}, then it is valid. If the result is \textit{sat}, then the returned kripke frame is a counter example that falsifies the original formula.

Since $NP??$ is the complexity of satisfiability of PDL formulas (discussed more in section \ref{sec:complexity}, then, it is possible for the solver to return \textit{unknown} as a result if the time limit is reached (default is 30 seconds). 
\section{Complexity of PDL formulas} \label{sec:complexity}

\section{Installation}

The following commands download the source code from github and compile it to generate the solver file ``pdl.jar". 

\begin{verbatim}
git clone https://github.com/mudathirmahgoub/pdl
cd pdl
chmod 777 gradlew
./gradlew build
cd bin
chmod 777 cvc4_linux
java -jar pdl.jar -i test.pdl 
java -jar plantuml.jar test.dot
\end{verbatim}



\section{Project components}

The project is primarily code and would use the SMT solver CVC4 as a back end and apply the relation theory to implement the semantics of PDL. The project would import CVC4 abstract syntax tree (AST) for relations from another project that I am working on (Alloy2SMT translator\footnote{\url{https://github.com/CVC4/org.alloytools.alloy/tree/cvc4/alloy2smt}}) which supports type checking and SMT models parsing, albeit it needs some refactoring to be more generic.  For this project I would write a translator from PDL AST to CVC4 AST and display back the SMT models returned from CVC4 as Kripke frames and dot files for visualization using software like graphvis.

Since CVC4 AST is written in Java, I am using Java for this project along with  gradle\footnote{\url{https://github.com/mudathirmahgoub/pdl}}. I have written an ANTLR4 grammar for PDL following the syntax in chapter 5 in \cite{dynamic}. The grammar also handles a kripke frame written using the set notation in chapter 5 with one difference ($m_{\mathfrak{K}}(a)$ would be written as $m(a)$). Lastly I  prepared classes for PDL AST for Kripke frames, formulas and programs. Next tasks would be parsing PDL input into PDL AST and translating this AST into CVC4 AST. 


\section{PDL formula translation}



\section{Examples}



\bibliographystyle{plain}

\bibliography{references}

\end{document}
