\documentclass[12pt,a4paper]{article}
\usepackage[utf8]{inputenc}
\usepackage{amsmath}
\usepackage{amsfonts}
\usepackage{amssymb}
\usepackage{makeidx}
\usepackage{graphicx}
\usepackage{lmodern}
\usepackage{url}
\usepackage{bm}
\usepackage{booktabs}
\usepackage[left=2cm,right=2cm,top=2cm,bottom=2cm]{geometry}
\author{Mudathir Mahgoub}
\title{Project proposal}
\begin{document}

\maketitle

\section{Propositional dynamic logic solver}
The idea of the project is to implement a solver for propositional dynamic logic (PDL). The input would be a formula in PDL and optionally a kripke structure for that formula. The output result is either \textit{unsat}, \textit{sat} or \textit{unknown}. If a kripke structure is given, the result scope would be restricted to this kripke structure. Otherwise, the result scope would be all possible kripke structures. 

\section{Implementation}
The project is primarily code and would use the SMT solver CVC4 as a back end and apply the relation theory to implement the semantics of PDL. The project would import CVC4 abstract syntax tree (AST) for relations from another project that I am working on (Alloy2SMT translator\footnote{\url{https://github.com/CVC4/org.alloytools.alloy/tree/cvc4/alloy2smt}}) which supports type checking and SMT models parsing. For this project I would write a translator from PDL AST to CVC4 AST and display back the SMT models returned from CVC4 as Kripke structures and dot files for visualization using software like graphvis.

\section{Progress so far}

Since the imported CVC4 AST is written in Java, I have created a java project for PDL  using gradle\footnote{\url{https://github.com/mudathirmahgoub/pdl}}. I have written an ANTLR4 grammar for PDL following the syntax in chapter 5 in \cite{dynamic}. Lastly I  prepared classes for PDL AST for formulas and programs. Next task is to integrate all these together and finish the implementation. 

\section{Preferences for presentation day}
I prefer April 30 for my presentation day.

\bibliographystyle{plain}

\bibliography{references}

\end{document}
